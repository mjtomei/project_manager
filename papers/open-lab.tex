\documentclass[11pt]{article}
\usepackage[inner=2.5cm, top=2.5cm, bottom=2.5cm, outer=4cm, marginparwidth=3cm]{geometry}
\geometry{letterpaper}

\usepackage{float, graphicx, caption, amssymb}
\usepackage[numbers]{natbib}
\usepackage[dvipsnames]{xcolor}
\usepackage{tabulary, booktabs}
\usepackage{fancyhdr}
\usepackage[scaled]{helvet}
\renewcommand*\familydefault{\sfdefault}

\usepackage[ddmmyyyy,hhmmss]{datetime}

\definecolor{tealgreen}{RGB}{0, 102, 102}
\definecolor{midnightgreen}{RGB}{0, 64, 77}

\usepackage{titlesec}
\titleformat{\section}
{\color{midnightgreen}\normalfont\Large\bfseries}
{\color{midnightgreen}\thesection}{1em}{}

\titleformat{\subsection}
{\color{tealgreen}\normalfont\large\bfseries}
{\color{tealgreen}\thesubsection}{1em}{}

\titleformat{\subsubsection}
{\color{tealgreen}\normalfont\normalsize\bfseries}
{\color{tealgreen}\thesubsubsection}{1em}{}

\usepackage{vhistory}
\usepackage{enumitem}
\usepackage{hyperref}
\hypersetup{
    colorlinks=true,
    linkcolor=midnightgreen,
    citecolor=tealgreen,
    urlcolor=tealgreen
}

\setlength{\parindent}{0pt}
\setlength{\parskip}{0.8em plus 0.1em minus 0.2em}

\pagestyle{fancy}
\fancyhead{}
\fancyfoot{}
\setlength{\headheight}{14pt}
\fancyhead[R]{\textcolor{gray}{\small Engineering Management}}
\fancyfoot[R]{\thepage}
\fancyfoot[C]{}
\fancyfoot[L]{}

\newcounter{note}
\stepcounter{note}

\renewcommand{\abstractname}{Summary}

\newcommand{\Note}[1]{%
    \marginpar{%
        \tiny{\color{tealgreen}{\thenote\ #1}}%
    }%
    \stepcounter{note}%
}

\newcommand{\MNote}[1]{%
    \marginpar{%
        \tiny{\color{tealgreen}{#1}}%
    }%
}

% Code/directory listing style
\usepackage{listings}
\lstset{
    basicstyle=\ttfamily\small,
    breaklines=true,
    frame=none,
    xleftmargin=1em,
    aboveskip=0.5em,
    belowskip=0.5em
}

\begin{document}

\title{\textcolor{midnightgreen}{Engineering Management}\\\large Open-Sourcing the Institution}
\author{}
\date{February 4, 2026}

\maketitle

\begin{abstract}
Organizational opacity is becoming unsustainable. Machine intelligence is
making institutional internals legible to outsiders whether institutions
cooperate or not. The costs of the opaque institutions we have --- excess
management, bureaucratic overhead, nepotism, lost productivity --- are
measured in trillions. Meanwhile, the cost of defending corporate secrets is rising sharply
while the cost of extracting them falls --- nation-state actors
demonstrated near-zero-marginal-cost espionage in the 2025 Notepad++
supply chain compromise, shifting the tradeoff against organizational
secrecy.

This paper proposes an alternative: open-source the institution itself.
Not just the research outputs, but the governance, resource flows,
decision history, and credit assignment. The concrete architecture ---
an organization repo with integrity checks, an automated recognition
system, and a recursive project management layer --- is implementable
with existing tools. The key insight is that integrity checks are to
organizational state what tests are to code: they hold the institution
accountable to its own stated rules.

Machine intelligence will make individual contribution legible whether institutions
cooperate or not. The choice is between a raw meritocracy where
legibility serves only those at the top, or a system of explicit
sharing built on honest data about who does what. Transparent
institutions are the infrastructure for the second option --- you
cannot build fair redistribution on opaque foundations. The tooling
described here is open source:
\url{https://github.com/mjtomei/project_manager}.
\end{abstract}

\clearpage

\tableofcontents
\clearpage

%% ============================================================
\section{Introduction}
\label{sec:intro}
%% ============================================================

Organizations are approaching a forced transition. Machine intelligence
is making institutional internals legible to outsiders whether
institutions cooperate or not. Nation-state actors are driving the cost of maintaining corporate
secrecy far above its competitive value --- the 2025
Notepad++ supply chain compromise ran undetected for six months, and the
tj-actions/changed-files attack exposed secrets across 23{,}000
repositories through a single compromised CI/CD
dependency. And intra-organization opacity results in its own 
inefficiencies --- excess management, bureaucratic bloat, nepotism
--- with costs measured in trillions. The question is no longer whether organizations will become
transparent, but whether they will manage the transition or have it
imposed on them.

This paper proposes managing it. We describe a research lab where the
lab itself is open source --- not just the research outputs, but the
organizational structure, decision-making processes, resource
allocation, and governance, all visible, forkable, and improvable by
anyone.

The core idea is that management becomes engineering. The difference is
that engineering tradeoffs are made explicit: documented, versioned,
debatable, auditable. Management tradeoffs currently live in the
discretion of whoever holds a management title, invisible to the people
affected by them. Making them explicit infrastructure --- resource
allocation rules, credit assignment criteria, accountability mechanisms
as code in a repo --- doesn't eliminate the tradeoffs. It makes them
visible so the people affected can participate in deciding them. Both
positive reinforcement (recognition, credit, visibility) and negative
reinforcement (integrity checks, conflict detection, staleness warnings)
become auditable systems rather than unaccountable discretion.

But the argument for transparency is not only about better governance.
Transparent institutions will reveal what opaque ones currently hide:
that contribution is unevenly distributed. This visibility is a
feature, not a bug. Opaque institutions obscure both freeloading and
unrecognized contribution --- they protect the people coasting and
punish the people carrying the load. As machine intelligence reshapes
the nature of work~\cite{amodei2026}, organizations will need to
decouple compensation from output, but you cannot do that honestly
without knowing what the output actually is. This is not a surveillance
argument. It is the observation that honest conversations about
contribution require honest data, and honest data requires transparent
systems.

%% ============================================================
\section{The End of Organizational Opacity}
\label{sec:opacity}
%% ============================================================

Machine intelligence is making organizational opacity increasingly
difficult to maintain. It can aggregate public filings, leaked
documents, employment records, financial data, and published research
into coherent pictures of what organizations are actually doing ---
regardless of what those organizations say they're
doing~\cite{fmi2024,kim2024}.

Organizations that don't open up voluntarily will have their opacity
pierced by others. Companies are deploying blockchain-based supply chain
verification because they can't verify what opaque suppliers claim about
sourcing and labor practices~\cite{marketus2024}. The era of ``trust us,
it works'' is ending across investment --- venture investors now conduct
spot-checks by contacting claimed customers directly and demanding bank
statements behind reported revenue, because they can no longer take
institutional self-reporting at face value~\cite{triunity2025}.
Self-insured employers are building internal claims analytics because
they can't trust insurer-reported numbers --- nearly half of insured
adults report receiving surprise medical bills, and those who dispute
denied claims get them reversed roughly half the
time~\cite{ajmc2024,aha2024}.

This is parallel construction at organizational scale --- building an
independent picture of an institution's internals from external data,
analogous to the law enforcement technique of reconstructing evidence
from public sources to avoid revealing classified intelligence methods.
When institutions are opaque, anyone with resources builds their own
apparatus to see through them. But it's not only the powerful who gain
this capability. The same tools that let investors reconstruct an
institution's internals are available to employees, researchers, and the
public. An individual can now aggregate an organization's public filings,
job postings, Glassdoor reviews, published outputs, and financial records
into a picture that used to require a dedicated analyst team. The opacity
that once protected mismanagement from scrutiny above also protected it
from scrutiny below. Both protections are ending at the same time.

The result is a world where the people running opaque organizations are
the last to know what everyone else --- funders, competitors, and their
own staff --- already sees. The question is not whether your organization
will become transparent. It is whether you will be the one who decides
how.

%% ============================================================
\section{The Shifting Economics of Corporate Secrecy}
\label{sec:secrets}
%% ============================================================

The previous section described how machine intelligence reconstructs
organizational internals from external data. But the case against
corporate secrecy is stronger than passive reconstruction: nation-state
actors are actively penetrating corporate infrastructure, and the
dynamics of this contest are shifting the cost-benefit tradeoff of
maintaining secrets decisively against the defender.

In June 2025, a Chinese state-sponsored hacking group compromised the
hosting infrastructure of Notepad++, an open-source text editor with
hundreds of millions of downloads~\cite{npp2026}. The attackers did not
exploit a vulnerability in the software itself. They gained access to
the shared hosting provider, intercepted update traffic, and selectively
redirected targeted users to malicious servers delivering a custom
backdoor codenamed Chrysalis~\cite{rapid72026}. The operation ran
undetected for six months. When Notepad++ creator Don Ho disclosed the
incident in February 2026, he could not even provide indicators of
compromise --- after a week analyzing 400~GB of server logs, his
incident response team found ``signs of an intrusion'' but ``no concrete
indicators of compromise --- such as binary hashes, domains, or IP
addresses''~\cite{npp2026}. The attackers were that clean.

Notepad++ is a text editor maintained by a small team. Consider what the
same capability implies when directed at a corporation's proprietary
infrastructure. If a nation-state actor can silently compromise a
software update mechanism for six months, it can read a corporation's
internal communications, source code, strategic plans, and trade secrets
at leisure. The Notepad++ case is not exceptional --- it is one instance
in a growing pattern of supply chain compromises including SolarWinds
(2020), Codecov (2021), 3CX (2023), and the tj-actions/changed-files
GitHub Action (2025)~\cite{tjactions2025}, each demonstrating the same
fundamental dynamic: infrastructure that organizations trust implicitly
becomes the attack vector.

Machine intelligence is accelerating this contest, but the acceleration
is asymmetric. Both attackers and defenders gain AI-assisted tools for
vulnerability discovery, code analysis, and anomaly detection. But the
attacker's problem is structurally easier: find one exploitable weakness.
The defender must protect every surface. As machine intelligence makes
both sides faster, the asymmetry does not shrink --- it compounds. A
system that can audit a million lines of code for vulnerabilities in
hours helps the defender, but it helps the attacker more, because the
attacker only needs one of those vulnerabilities to succeed.

This asymmetry also means that a corporation's home country defense
budget is irrelevant. A company headquartered in the country with the
world's largest military and intelligence apparatus is not thereby
protected from espionage by a smaller state. The defending nation would
need to secure every piece of infrastructure its corporations depend on
--- every hosting provider, every software dependency, every cloud
service, every third-party vendor. The attacking nation needs to find one
gap in that chain. The Notepad++ attackers did not breach a hardened
government target; they compromised a shared hosting server running a
PHP script. The supply chain extends far beyond what any defense
establishment monitors or controls.

The current state of corporate infrastructure makes this worse than the
asymmetry alone would suggest. Most organizations' software supply
chains were never designed to resist nation-state attackers. Update
mechanisms lack cryptographic verification. Dependencies are pulled from
public registries without integrity checks. Hosting infrastructure is
shared. Internal networks assume perimeter security. These are not
obscure vulnerabilities --- they are standard practice. Notepad++ lacked
certificate verification in its updater until version 8.8.8, released in
November 2025~\cite{npp2026}. This was not negligence; it was normal.
The tj-actions/changed-files incident in March 2025 demonstrated the
same pattern in CI/CD infrastructure: an attacker compromised a single
GitHub Action used by over 23{,}000 repositories, injecting code that
dumped CI runner secrets --- AWS access keys, GitHub tokens, private RSA
keys --- into publicly readable workflow
logs~\cite{tjactions2025}. The attack cascaded through a chain of
dependent actions (reviewdog/action-setup $\to$ tj-actions/eslint-changed-files
$\to$ tj-actions/changed-files), illustrating how a single compromised
dependency propagates through the ecosystem. This is the infrastructure
that corporations rely on to build and deploy their proprietary software.
The entire ecosystem is unprepared, which means there will be many entry
points for sophisticated attackers, at least for the foreseeable future
while the industry catches up.

The implication for organizational opacity is direct: the cost of
defending corporate secrets is rising rapidly while the value of those
secrets --- relative to what an attacker can extract --- is falling.
Once automated espionage tooling is built and deployed, the marginal
cost of targeting an additional organization approaches zero. The
traditional justification for corporate opacity --- that secrecy
protects trade secrets and competitive position --- assumed a world
where keeping secrets was cheap relative to their value. That tradeoff
is inverting for any organization of sufficient interest to a state
actor.

Open organizations shift competitive advantage to ground that is not
vulnerable to this dynamic. When your governance, processes, and
institutional knowledge are public by design, the attack surface for
espionage shrinks dramatically. The competitive advantage shifts to
execution speed, talent density, and institutional culture --- none of
which can be stolen by compromising a hosting provider. This is not an
argument that all information should be public. It is the observation
that the set of information worth keeping secret is much smaller than
most organizations assume, and the cost of defending even that smaller
set against state-level adversaries is higher than most organizations
realize.

%% ============================================================
\section{The Cost of Opaque Institutions}
\label{sec:cost}
%% ============================================================

The economic costs of organizational opacity are substantial, though
precise measurement is difficult. The estimates in
Table~\ref{tab:costs} draw from a mix of peer-reviewed research,
industry surveys, and trade publications of varying methodological
rigor.

In one Australian study, non-administrative staff reported spending 6.4
hours per week --- 16\% of a standard work week --- complying with
internal regulation, and the cost of self-imposed internal rules
exceeded the cost of government regulation by more than two to
one~\cite{deloitte}.

Nepotism and favoritism compound these costs. Nearly 75\% of employees
report having worked in a toxic workplace, with poor leadership ---
including favoritism and lack of accountability --- cited as the top
cause by 79\% of them~\cite{ihire2025}. Toxic culture is 10 times more
important than compensation in predicting
turnover~\cite{sull2022b}. When unqualified hires hold leadership roles
through connections rather than competence, the remaining staff pick up
the slack, leading to burnout and further turnover. The organizational
knowledge needed to actually run the institution is never written down
--- it lives in the heads of whoever happens to be politically connected
enough to stay.

The same patterns appear in public institutions. Researchers receiving
federal grants spend over 44\% of their time on administrative tasks
rather than research --- a figure that has remained stubbornly consistent
across FDP surveys in 2005, 2012, and
2018~\cite{fdp2018}. Nepotism in public institutions creates wasteful
overstaffing where unneeded positions are created to employ relatives ---
documented across municipalities as a drag on economic
development~\cite{nepotism2020}. Perceived nepotism is negatively
associated with educational investment across countries, as measured by
PISA scores~\cite{nepotism_hc2020}. Funding decisions happen behind
closed doors. Credit assignment is political. Access to resources depends
on who you know, not what you contribute.

These costs become especially dangerous heading into machine-intelligence-driven economic
disruption. If we cannot see who contributes what inside our
institutions, we have no foundation for the redistribution that machine-intelligence
productivity gains will require. Amodei~\cite{amodei2026} argues that
society will need to pay people even when they are not providing
traditional economic value. That is not possible to do fairly --- or
even to do at all --- without transparent systems that make
contribution legible in the first place.

\begin{table}[ht]
\centering
\small
\begin{tabulary}{\textwidth}{LRl}
\toprule
\textbf{Cost} & \textbf{Estimate} & \textbf{Source} \\
\midrule
Excess management \& bureaucracy (U.S.) & \$3T/yr (\texttildelow17\% GDP) & \cite{hamel2016} \\
Fortune 500 back-office inefficiencies & \$480B/yr & \cite{ssonetwork} \\
Regulatory compliance (U.S.\ aggregate) & \$103--289B/yr & \cite{cato} \\
Toxic workplace culture (U.S.) & \$50B/yr & \cite{sull2022b} \\
Hospital claims overturn spending & \$19.7B (2022) & \cite{aha2024} \\
Unreimbursed university overhead & \$6.8B/yr & \cite{aau} \\
\midrule
\multicolumn{3}{l}{\textit{Markets responding to opacity}} \\
Blockchain supply chain verification & \$2.9B $\to$ \$44.3B by 2034 & \cite{marketus2024} \\
Open-source intelligence (OSINT) & \$15B $\to$ \$49B by 2029 & \cite{fmi2024} \\
\bottomrule
\end{tabulary}
\caption{Estimated costs of organizational opacity and markets emerging
to circumvent it. Figures are U.S.-specific unless noted.}
\label{tab:costs}
\end{table}

%% ============================================================
\section{Open-Sourcing the Institution}
\label{sec:open}
%% ============================================================

Most ``open science'' initiatives open-source the outputs (papers, data,
code) but keep the institution opaque. This lab open-sources the
institution itself: governance, resource flows, decision history, credit
assignment. The structure is the product.

%% ============================================================
\section{Architecture}
\label{sec:arch}
%% ============================================================

Here is what it looks like when an organization has no secrets from
itself. Higher-level units (the lab itself, divisions within it) have
two layers of state: an organization repo and a project management
layer (managed by \texttt{pm}). Individual research projects below them
may only need the pm layer and their code repo. The organizational layer
is for units that need to track more than just a tech tree.

\subsection{The Organization Repo}
\label{sec:org-repo}

A GitHub repo that is the public record of the organization. This is
where things live that don't belong in any individual project but need to
be tracked: documents, small scripts, spreadsheets, test programs,
analyses, meeting notes, governance decisions, and anything else that
should be visible and version-controlled but doesn't need wide release as
a standalone project.

It also includes pointers to the other repos and projects in the
organization --- individual project repos, external dependencies --- in
\texttt{docs/} so it can act as a central dispatch.

\begin{lstlisting}
org-repo/
|-- docs/                        # golden copy of project state
|   |-- governance/              # decision-making processes, roles
|   |-- research-agenda/         # what the lab works on and why
|   |-- resources/               # grant tracking, allocation records
|   `-- onboarding/              # how to join, how things work
|-- members/
|   |-- alice/                   # Alice's working directory
|   |   |-- notes/               # her meeting notes, scratch work
|   |   |-- scripts/             # one-off analysis scripts
|   |   `-- proposals/           # drafts she's working on
|   |-- bob/
|   |   |-- notes/
|   |   |-- data/                # small datasets, spreadsheets
|   |   `-- experiments/         # test programs, prototypes
|   `-- ...
|-- checks/                      # integrity checks (see below)
`-- archive/                     # completed work, historical records
\end{lstlisting}

\textbf{Member directories.} Every member gets their own directory ---
their workspace within the org. They can put anything there: notes,
scripts, Excel sheets, draft proposals, experimental code. Their work is
visible to the rest of the organization without requiring them to publish
it anywhere else.

When a member wants to promote something to the project-wide golden copy,
they open a PR moving or copying it into \texttt{docs/}. The PR is a
request for the organization to recognize and adopt the work. Other
members review it, and the merge is the organization saying ``yes, this
is now part of our shared state.''

\textbf{The \texttt{docs/} directory.} The golden copy --- the
authoritative state of the project. Nothing gets into \texttt{docs/}
without a PR. The PR history in \texttt{docs/} is the lab's
institutional memory.

\textbf{The \texttt{checks/} directory.} Automated integrity checks that
project managers create and maintain. These run against the repo (via CI
or on-demand) and surface issues that humans should look at.

\subsection{Integrity Checks}
\label{sec:checks}

Integrity checks are to organizational state what tests are to code.
When machine intelligence produces code, tests check that the code does what it
claims. When an organization produces documents, decisions, and resource
allocations, integrity checks do the same thing: they verify
organizational state against the organization's own stated rules and
surface inconsistencies for human judgment.

This applies to any organization that manages resources --- public or
private, academic or corporate. The checks maintain a history of both
automated and human-assisted audits. Every check run is logged. When a
human reviews an issue surfaced by a check and makes a judgment call,
that judgment is recorded too. Over time this builds an audit trail
showing not just the current state but how the organization has responded
to issues as they were discovered --- an accountability mechanism that
persists regardless of personnel changes.

Examples:

\begin{itemize}[leftmargin=*]
    \item \textbf{Consistency}: does the resource allocation in
    \texttt{docs/resources/} add up? Do budget numbers match grant
    amounts recorded in governance decisions?

    \item \textbf{Staleness}: are there proposals sitting for months
    with no PR to \texttt{docs/}? Governance documents referencing
    people who are no longer active?

    \item \textbf{Completeness}: does every active grant have an
    allocation record? Does every research project in the agenda have
    a corresponding pm instance?

    \item \textbf{Conflict}: contradictory statements across governance
    documents? Two proposals allocating the same funds differently?

    \item \textbf{Attribution}: does every work product have clear
    attribution? Documents referencing work without crediting the
    contributor?
\end{itemize}

These checks can be simple scripts, LLM-powered analyses, or structured
validators. They're not gates --- they surface issues for human judgment,
same as a test suite surfaces failures for a developer to evaluate. The
checks themselves are in the repo, subject to the same review process as
everything else.

\subsection{Automated Recognition}
\label{sec:recognition}

The same infrastructure that detects problems also recognizes
achievements. In traditional organizations, recognition flows through
managers, shaped by who is visible, who is in the right social circle,
who shares the manager's background. Automated recognition replaces this
with a read operation on organizational state that's already being
maintained.

Examples: milestone completion across a tech tree; a new member's first
PR to \texttt{docs/}; sustained contribution over time; a PR that
unblocks several downstream projects; a grant's deliverables all
completed with the full funding-to-output chain visible.

Because the org repo and pm state are readable text and
structured data, anyone can ask an LLM to assess contributions
holistically --- reading a member's directory, their PRs, the projects
they've touched, the discussions they've participated in. This is closer
to what a thoughtful colleague would say if they'd been paying attention
to everything, which no human can do at organizational scale.

This avoids the problems of fixed metrics. Lines of code, commit
frequency, and citation counts reward the metric instead of the work.
LLM-based assessment introduces its own risks --- LLMs have predictable
biases toward verbose, confident, and recently-active contributors, and
their evaluation can be anticipated and gamed. The game doesn't
disappear; it becomes a meta-game. But because the recognition criteria,
prompts, and outputs are all in the repo, anyone can see how the game is
being played. The meta-game itself is auditable --- if someone is
optimizing for the evaluator rather than doing useful work, that pattern
is visible in the same data the evaluator reads. You can write integrity
checks against the meta-game the same way you write them against the
organization's primary state.

The point is that the organization notices what its members accomplish
without requiring self-promotion or managerial attention. The system does
the noticing. People decide what to do with it. This is not a tool for
optimizing productivity --- it will surface what looks like inefficiency
but is actually useful exploration, and the organization should
understand that distinction.

\subsection{The Project Management Layer}
\label{sec:pm}

Managed by
\texttt{pm}\footnote{\texttt{pm} is an open-source CLI tool for managing
dependency graphs of work items across parallel contributors. Source and
documentation: \url{https://github.com/mjtomei/project_manager}}, the
project management layer maintains a \texttt{project.yaml} file,
\texttt{plans/} directory, and a dependency graph (the ``tech tree'')
of everything that needs to happen. \texttt{pm} tracks what needs to be
built, in what order, what's blocked, what's ready, and who's working on
what. It generates prompts for machine intelligence coding agents and detects when branches
merge, automatically unblocking downstream work. The pm state can live
inside the organization repo or in a standalone repository --- the
tooling is agnostic.

The pm layer integrates with a recursive tech tree system. The lab's
top-level tree connects the organizational layer to all the research
projects below it. In \emph{prescriptive} mode, a parent tree suggests
work to child projects (the child opts in and can accept or decline). In
\emph{descriptive} mode, a parent tree observes child projects without
directing them --- a read-only aggregation that requires no coordination.

The organization repo and the pm layer serve separate concerns: the org
repo is the record of what the organization is and has done; the pm
layer is the plan for what it's doing next. Setting up a new
organization requires initializing both: \texttt{pm init} for the pm
layer, and creating the org repo with the directory structure described
in Section~\ref{sec:org-repo}. A walkthrough demonstrating the full
setup --- from an empty repo to a working organization with integrity
checks, member directories, and a populated tech tree --- is forthcoming
as a companion to this paper.

The pm layer will expose what looks like inefficiency --- blocked work,
stalled plans, idle projects. Some of that is real inefficiency. Some of
it is useful exploration. The visibility lets the organization have
honest conversations about what's happening.

%% ============================================================
\section{Relationship to Existing Institutions}
\label{sec:existing}
%% ============================================================

The lab is additive, not exclusionary. Researchers can participate while
holding positions at existing institutions. Grants can come from
traditional funders and flow through traditional fiscal sponsors. The
transparency is about what happens with the resources, not about
requiring new channels.

As long as existing power structures persist, people will interface with
governments through existing institutions for taxes, employment law, visa
sponsorship, and compliance. A researcher receiving a stipend still files
taxes through their employer. A grant still goes through a fiscal
sponsor. The open layer sits on top of these structures, making decisions
and flows visible without pretending the underlying structures don't
exist.

This also matters for funders. All else equal, an organization where a
machine intelligence can read and verify what's actually happening is
less risky to fund than an opaque one. As this kind of transparency
becomes feasible, it will increasingly be expected --- not because anyone
mandates it, but because the alternative becomes an obviously worse bet.

%% ============================================================
\section{Limitations and Open Questions}
\label{sec:limitations}
%% ============================================================

\textbf{The transparency paradox.} Bernstein~\cite{bernstein2012}
demonstrated that factory workers were more productive when given privacy
from management observation. This is a genuine finding, but the
mechanism matters: Bernstein's workers had no control over the
observation system and no ability to use it themselves. Transparency was
asymmetric --- management watched workers, not the reverse. In that
configuration, transparency is surveillance, and surveillance induces
hiding behavior. This project is specifically an attempt at solving the
problem Bernstein identified, by making transparency symmetric. When
every member has the same machine-intelligence-assisted access to
organizational state that executives and project managers have, the power
asymmetry that drives hiding behavior is substantially reduced. A tech
worker with machine intelligence tools can read governance documents, audit resource flows,
and assess organizational health with nearly the same capability as the
people nominally in charge. The mechanism described here is more
symmetric than prior transparency systems precisely because machine
intelligence makes organizational legibility available to everyone, not
just to those with management titles. Nevertheless, the risk of chilling
effects on exploratory work remains real. Members might avoid speculative
projects that could look unproductive. The member directories are
designed to mitigate this --- they are personal workspaces where
incomplete and exploratory work is expected --- but the tension between
organizational legibility and individual creative freedom requires
ongoing attention.

\textbf{Legitimate uses of opacity.} Not all organizational opacity is
pathological. Negotiating positions, personnel matters, and shielding
individuals from external pressure are cases where some degree of
opacity serves legitimate purposes. The architecture described here does
not require total transparency --- the org can choose what is
world-readable and what is member-only. But the paper's argument is
strongest for governance, resource allocation, and credit assignment,
where opacity more often enables abuse than serves legitimate ends. The
traditional case for trade secret protection is further weakened by the
analysis in Section~\ref{sec:secrets}: as the cost of defending secrets
against state-level adversaries rises, the net competitive value of
maintaining them falls.

\textbf{LLM gaming and bias.} The automated recognition system inherits
the risks of algorithmic management documented by
Kellogg~et~al.~\cite{kellogg2020}. LLMs have predictable biases and
their assessments can be gamed by those who understand the model's
preferences. Making the recognition infrastructure open and auditable means the
meta-game --- gaming the evaluator rather than doing the work --- is
itself visible and auditable. This does not eliminate strategic behavior,
but it makes each layer of strategy legible to anyone who cares to look.
Formal analysis of these recursive gaming surfaces is future work.

\textbf{No empirical validation.} This paper describes an architecture,
not empirical results. The claims about cost reduction, improved
fairness, and organizational health are predictions, not measurements. A
pilot deployment (even at small scale: 5--10 people for 3--6 months)
measuring administrative overhead, perceived fairness, and contribution
distribution would substantially strengthen the argument. Formal
modeling --- agent-based simulation of organizational dynamics under
transparent vs.\ opaque regimes, game-theoretic analysis of strategic
behavior, principal-agent models adapted to symmetric transparency ---
is also future work. This is a design proposal; the empirical and
theoretical validation comes next.

\textbf{Cost figures.} The economic estimates in Section~\ref{sec:cost}
draw from sources of varying rigor. They
indicate the scale of the problem but should not be treated as precise
measurements. In particular, the \$3 trillion figure from
Hamel~\cite{hamel2016} is based on extrapolation from exemplar companies,
not a comprehensive economic study.

%% ============================================================
\section{Related Work}
\label{sec:related}
%% ============================================================

The idea of making organizational governance explicit and transparent has
a substantial history. This section positions the present work relative
to prior approaches and identifies what is genuinely new.

\textbf{Commons governance.} Ostrom~\cite{ostrom1990} established eight
design principles for self-governing institutions managing shared
resources: clearly defined boundaries, proportional costs and benefits,
collective choice arrangements, monitoring, graduated sanctions, conflict
resolution mechanisms, minimal recognition of rights to organize, and
nested enterprises. The architecture described in this paper instantiates
several of these principles --- integrity checks as monitoring,
governance-as-code as collective choice arrangements, the recursive tech
tree as nested enterprises --- in a version-controlled, machine-readable
form. Ostrom's framework provides the theoretical grounding; we provide a
concrete implementation using modern software infrastructure.

\textbf{Self-management frameworks.} Holacracy~\cite{robertson2015}
encodes governance rules explicitly, replacing traditional management
hierarchy with a constitution that defines roles, accountabilities, and
decision-making processes. Laloux~\cite{laloux2014} surveys organizations
operating with radical transparency, including Buurtzorg and Morning
Star. Our approach shares the goal of making governance rules explicit
but differs in two ways: the rules are stored in Git (not a proprietary
constitution), making them forkable and diffable; and machine
intelligence is used to check organizational state against the rules,
rather than relying solely on human process adherence. Holacracy's
adoption difficulties --- Zappos being the most prominent case --- also
illustrate that rigid governance frameworks can fail when they don't
accommodate organizational ambiguity. Our approach is deliberately
flexible: governance documents are prose, not a fixed schema.

\textbf{Decentralized Autonomous Organizations (DAOs).} DAOs encode
governance in smart contracts on a blockchain, making institutional rules
machine-readable and forkable~\cite{rozas2021}. The goals overlap
significantly with ours. The key differences are infrastructural: our
approach uses Git, which is universally understood by developers, does
not require tokens or chain infrastructure, and integrates with existing
development tools. DAOs also have a well-documented history of governance
failures --- including the 2016 DAO exploit, where transparent
machine-readable rules were used against the organization, and
persistent voter apathy in token-based governance. We take these as
evidence that governance transparency is necessary but not sufficient;
the integrity checks and recognition system described in
Section~\ref{sec:arch} are designed to address some of these failure modes.

\textbf{Open-source governance.} The open-source software movement
provides decades of evidence on transparent organizational
structures~\cite{raymond1999}. Research on open-source governance shows
both the power of open contribution models and their failure modes:
hidden hierarchies, burnout among maintainers, and difficulty with
accountability when no one is formally in charge. Buffer's radical
transparency experiment (public salaries and open financials since 2013)
provides real-world data on organizational transparency in a corporate
setting. Valve's nominally flat structure produced documented problems
with clique-based resource allocation despite apparent openness --- a
cautionary tale that motivates our integrity checks, which are designed
precisely to surface hidden power structures that persist even under
nominal transparency.

\textbf{The transparency paradox.} Bernstein~\cite{bernstein2012} found
that factory workers were more productive when given privacy from
management observation, showing that transparency can reduce performance
by inducing hiding behavior. This is an important counterpoint, but the
mechanism matters: Bernstein's workers had no control over the
transparency system and no ability to use it themselves. The transparency
was asymmetric --- management watched workers, not the reverse. This
project is specifically an attempt at solving that problem by making
transparency symmetric: machine intelligence gives every member the same
ability to read and audit organizational state that was previously
reserved for management. The remaining risk --- that even symmetric
transparency produces chilling effects on exploratory work --- is
addressed in Section~\ref{sec:limitations}.

\textbf{Algorithmic management.} Kellogg et~al.~\cite{kellogg2020}
provide a systematic review of algorithmic management in the workplace,
documenting how automated systems can reproduce or amplify existing
power asymmetries. The automated recognition system described in
Section~\ref{sec:recognition} is an instance of algorithmic management
and inherits these risks. We address this by making the recognition
infrastructure itself open and auditable --- the criteria are code, not
black-box algorithms --- but acknowledge that this does not eliminate all
risks (see Section~\ref{sec:limitations}).

%% ============================================================
\section{Conclusion}
\label{sec:conclusion}
%% ============================================================

Organizations rely on opacity --- implicit systems of credit, funding,
access, and decision-making understood by insiders but invisible to
outsiders and unaccountable to anyone. Machine intelligence is ending
this arrangement. Machine intelligence already reconstructs what institutions are
actually doing regardless of what those institutions claim. This paper
argues that organizations face a choice: adopt structured transparency
on their own terms, or have it imposed from outside. We describe an
architecture for \emph{open-sourcing the institution itself} ---
governance, resource flows, decision history, and credit assignment ---
using version-controlled repositories, automated integrity checks, and
machine-readable organizational state. Management becomes engineering:
both positive reinforcement (recognition, credit, visibility) and
negative reinforcement (integrity checks, conflict detection, staleness
warnings) are code in the repository, auditable by anyone. We present
the design of a research lab built on these principles and discuss its
applicability to any organization that manages resources and produces
work products.

%% ============================================================
%% Bibliography
%% ============================================================

\clearpage
\begin{thebibliography}{99}

\bibitem{ostrom1990}
E.~Ostrom, \emph{Governing the Commons: The Evolution of Institutions
for Collective Action}. Cambridge University Press, 1990.

\bibitem{robertson2015}
B.~J.~Robertson, \emph{Holacracy: The New Management System for a
Rapidly Changing World}. Henry Holt, 2015.

\bibitem{laloux2014}
F.~Laloux, \emph{Reinventing Organizations}. Nelson Parker, 2014.

\bibitem{rozas2021}
D.~Rozas, A.~Tenorio-Forn\'{e}s, S.~Djunadi, and J.~Hassan,
``Analysis of the Potentials of Blockchain Technology for Commons
Governance,'' 2021.

\bibitem{raymond1999}
E.~S.~Raymond, \emph{The Cathedral and the Bazaar}. O'Reilly, 1999.

\bibitem{bernstein2012}
E.~S.~Bernstein, ``The Transparency Paradox: A Role for Privacy in
Organizational Learning and Operational Control,''
\emph{Administrative Science Quarterly}, vol.~57, no.~2, pp.~181--216,
2012.

\bibitem{kellogg2020}
K.~C.~Kellogg, M.~A.~Valentine, and A.~Christin, ``Algorithms at
Work: The New Contested Terrain of Control,'' \emph{Academy of
Management Annals}, vol.~14, no.~1, pp.~366--410, 2020.

\bibitem{marketus2024}
Market.us, ``Blockchain for Supply Chain Traceability Market Size,'' 2024.
\url{https://market.us/report/blockchain-for-supply-chain-traceability-market/}

\bibitem{triunity2025}
Triunity Capital Ventures, ``Due Diligence in 2025: What Investors
Really Want to Know About Your AI Startup.''
\url{https://triunitycapitalventures.com/due-diligence-in-2025-what-investors-really-want-to-know-about-your-ai-startup/}

\bibitem{ajmc2024}
AJMC, ``Survey Exposes Pervasive Billing Errors, Aggressive Tactics in
US Health Insurance,'' 2024.
\url{https://www.ajmc.com/view/survey-exposes-pervasive-billing-errors-aggressive-tactics-in-us-health-insurance}

\bibitem{hamel2016}
G.~Hamel, ``Excess Management Is Costing the U.S.\ \$3~Trillion Per
Year,'' \emph{Harvard Business Review}, 2016.
\url{https://hbr.org/2016/09/excess-management-is-costing-the-us-3-trillion-per-year}

\bibitem{ssonetwork}
SSO Network, ``Fortune 500: \$480~Billion in Back Office Inefficiencies
a Year.''
\url{https://www.ssonetwork.com/continuous-improvement-process-improvement/articles/480-billion-in-back-office-inefficiencies-how-to}

\bibitem{cato}
NBER / Cato, ``The Cost of Regulatory Compliance in the United States.''
\url{https://www.cato.org/research-briefs-economic-policy/cost-regulatory-compliance-united-states}

\bibitem{deloitte}
Deloitte Australia, ``Get Out of Your Own Way --- Unleashing
Productivity,'' cited in \emph{Consultancy.uk}.
\url{https://www.consultancy.uk/news/973/deloitte-compliance-costs-australian-economy-250-billion}

\bibitem{ihire2025}
iHire, ``Toxic Workplace Trends Report,'' February 2025 ($n=2{,}285$
across 57 industries).
\url{https://www.ihire.com/about/press/ihire-toxic-workplace-trends-report-pr}

\bibitem{sull2022b}
D.~Sull, C.~Sull, W.~Cipolli, and C.~Brighenti, ``Why Every Leader
Needs to Worry About Toxic Culture,'' \emph{MIT Sloan Management Review},
March 2022. \$50B figure originally from SHRM, ``The High Cost of a
Toxic Workplace Culture,'' 2019.
\url{https://sloanreview.mit.edu/article/why-every-leader-needs-to-worry-about-toxic-culture/}

\bibitem{fdp2018}
S.~L.~Schneider, ``2018 FDP Faculty Workload Survey: Research Report:
Primary Findings,'' Federal Demonstration Partnership Foundation, 2020.
\url{https://thefdp.org/wp-content/uploads/FDP-FWS-2018-Primary-Report.pdf}

\bibitem{aau}
Association of American Universities, ``Frequently Asked Questions about
Facilities and Administrative Costs of Federally Sponsored University
Research.''
\url{https://www.aau.edu/key-issues/frequently-asked-questions-about-facilities-and-administrative-costs}

\bibitem{nepotism2020}
``Nepotism, Political Competition and Overemployment,''
\emph{Political Research Exchange}, 2020.
\url{https://www.tandfonline.com/doi/full/10.1080/2474736X.2020.1781542}

\bibitem{nepotism_hc2020}
``Nepotism, Human Capital and Economic Development,''
\emph{Journal of Economic Behavior \& Organization}, 2020.
\url{https://www.sciencedirect.com/science/article/abs/pii/S0167268120304431}

\bibitem{aha2024}
American Hospital Association, ``Skyrocketing Hospital Administrative
Costs: Burdensome Commercial Insurer Policies Are Impacting Patients and
the Health Care Workforce,'' September 2024.
\url{https://www.aha.org/guidesreports/2024-09-10-skyrocketing-hospital-administrative-costs-burdensome-commercial-insurer-policies-are-impacting}

\bibitem{amodei2026}
D.~Amodei, ``The Adolescence of Technology,'' 2026.
\url{https://darioamodei.com/essay/the-adolescence-of-technology}

\bibitem{fmi2024}
Future Market Insights, ``Open Source Intelligence (OSINT) Market
Size, Share \& Trends,'' 2024.
\url{https://www.futuremarketinsights.com/reports/open-source-intelligence-market}

\bibitem{kim2024}
A.~Kim, M.~Muhn, and V.~Nikolaev, ``Financial Statement Analysis with
Large Language Models,'' \emph{Chicago Booth Research Paper}, 2024.
\url{https://bfi.uchicago.edu/working-paper/financial-statement-analysis-with-large-language-models/}

\bibitem{tjactions2025}
Wiz Threat Research, ``GitHub Action tj-actions/changed-files Supply
Chain Attack (CVE-2025-30066),'' March 2025.
\url{https://www.wiz.io/blog/github-action-tj-actions-changed-files-supply-chain-attack-cve-2025-30066}

\bibitem{npp2026}
D.~Ho, ``Notepad++ Hijacked by State-Sponsored Hackers,'' February 2026.
\url{https://notepad-plus-plus.org/news/hijacked-incident-info-update/}

\bibitem{rapid72026}
Rapid7, ``Notepad++ Hosting Breach Attributed to China-Linked Lotus
Blossom Hacking Group,'' reported by \emph{The Hacker News}, February
2026.
\url{https://thehackernews.com/2026/02/notepad-hosting-breach-attributed-to.html}

\end{thebibliography}

\end{document}
